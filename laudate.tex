\documentclass[oneside, 10pt]{article}
\usepackage[margin=3.5cm]{geometry}
\usepackage{fontspec}
\setmainfont[Mapping=tex-text]{Linux Libertine O}
\usepackage[latin, french]{babel}
\usepackage{at}
\usepackage{textcomp}
\usepackage{url}
\usepackage{verse}
\usepackage{tocbibind}
\usepackage{multind}
\usepackage{hyperref}
\usepackage{color}
\usepackage{xcolor}
\usepackage{fontspec}
\usepackage{pdfcolparallel}
\usepackage{paralist}
\usepackage{parcolumns}
\usepackage{lettrine}
\usepackage{yfonts}
\usepackage{multind}
\usepackage[rm,sc,big,center]{titlesec}
\usepackage{gregoriotex}
\usepackage{graphicx}
\titlelabel{}
\titleformat*{\section}{\LARGE\scshape\centering}

\setromanfont[Mapping=tex-text, Numbers=OldStyle, Ligatures=Historical, Alternate=1]{Linux Libertine O}
\newfontfamily\freeserif{FreeSerif}
\newfontfamily\biolinum{Linux Biolinum O}
\newcommand{\enluminure}[2]{\lettrine[lines=2]{\small \initfamily #1}{#2}}
\def\greinitialformat#1{%
{\fontsize{43}{43}\selectfont #1}%
}


\newcommand{\Respons}{{\freeserif \char"211F}}
\newcommand{\Versicle}{{\freeserif \char"2123}}

% fontes
\newcommand{\fontlat}[1]{\biolinum #1}
\newcommand{\fontfr}[1]{#1}

\newcommand{\SECTION}[1]
{
  \section*{#1}
  \addcontentsline{toc}{section}{#1}
}

\newcommand{\SUBSECTION}[1]
{
  \subsection*{#1}
  \addcontentsline{toc}{subsection}{#1}
}

%%%% PARALLEL %%%

\newcommand{\BEGINPARA}{\begin{Parallel}[c]{0.46\textwidth}{0.46\textwidth}}
\newcommand{\LR}[2]{\selectlanguage{french}\ParallelLText{\fontlat{#1}}
                   \selectlanguage{french}\ParallelRText{\fontfr{#2}}\ParallelPar}
\newcommand{\ENDPARA}{\end{Parallel}}

\newcommand{\BEGINENLUL}{\noindent\begin{minipage}[t]{0.44\textwidth}}
  \newcommand{\BEGINENLUR}{\noindent\begin{minipage}[t]{0.51\textwidth}}
    \newcommand{\ENDENLU}{\end{minipage}}

%% insertion grégorien
\newcommand{\gregorien}[3]{
\gresetfirstlineaboveinitial{\small \textsc{\textbf{#1}}}{\small \textsc{\textbf{~#1}}}
\includescore{#2}
}

% TITRE

\makeatletter
\def\maketitle{
  \begin{titlepage}
    \begin{center}
      {\Large \@author}
      \null\vfill
    \end{center}
    \begin{center}
      {\huge \@title}\\
      \vskip 0.5cm
      {\Large \emph{\@subtitle}}
      % \vskip 1cm
      % {\texttt{\@email}}
      \vskip 2cm
      % {\Large \@date}
      {\normalsize {\it
          1\iere{} édition.}}
      \vfill\null
      {\small Creative Commons BY-NC-SA}
    \end{center}
  \end{titlepage}
}
\def\email#1{\def\@email{#1}}
\def\subtitle#1{\def\@subtitle{#1}}
\makeatother


\title{Laudate Dominum}
\subtitle{Proposition de réintroduction des \emph{Laudes} pour la Liturgie des Heures rénovée.}
\author{\textsc{Jean-baptiste Bourgoin}}
\email{jeanbaptiste.bourgoin@gmail.com}
\date{2013}

\begin{document}

\maketitle

% #############################################################################

\tableofcontents

\newpage
\SECTION{Introduction}

Les \emph{laudes}, le regroupement des psaumes 148 à 150 qui a donné son nom à la prière du matin de l'Office Romain et que l'on retrouve dans la grande majorité des liturgies chrétiennes antiques (aussi bien orientales qu'occidentales), ont disparues en tant que telles du psautier romain depuis la réforme du bréviaire de saint Pie X en 1911.

La suppression des \emph{laudes} lors de la réforme de saint Pie X fut vivement critiquée par certains liturgistes tels que Anton baumstark :

\begin{quote}
Hence to the reformers of the \emph{Psalterium Romanum} belongs the distinction of having brought to an end the universal observance of a liturgical practice which was followed, one can say, by the Divine Redeemer Himself during His life on earth.\footnote{Anton Baumstark, \emph{Comparative Liturgy}, 1958, cité par Paul Cavendish dans \emph{An Introduction to the Reform of the Roman Breviary 1911-13}}
\end{quote}

Mais les \emph{laudes} furent préservées dans l'Office Monastique qui a toujours conservé l'\textit{ordo psallendi} de saint Benoît\footnote{Aujourd'hui les communautés bénédictines célébrant selon la forme ordinaire peuvent choisir entre l'\textit{ordo psallendi} de saint Benoit et celui de Paul VI}.

Aujourd'hui la grande souplesse de la liturgie rénovée par Paul VI nous permet la réintroduction des \emph{laudes} dans l'Office du Matin.

C'est ce que propose ce petit document après avoir présenté à partir de tableaux la place des \emph{laudes} dans les liturgies latines et dans la liturgie romaine plus particulièrement.

Ce document est \textbf{réalisé dans l'esprit de \textit{Summorum Pontificum}} et l'appel du pape Benoit XVI a faire s'enrichir mutuellement les deux formes du rite romain.

Ce document est \textbf{soucieux de ne pas sortir des règles établies dans la \textit{Présentation Générale de la Liturgie des Heures}} afin de préserver à l'Office rénové sa forme et son caractère.

Ce document s'appuie sur les recherches historiques, telles que celles de Robert Taft dans la \textit{Liturgie des Heures en Orient et en Occident}, afin de proposer une redécouverte sérieuse d'éléments antiques de la prière chrétienne.

Sur ce point, la réforme de Paul VI est admirable : réintroduction des Preces ; du Notre Père aux grandes heures ; multiplication substantielle des pièces (antiennes multipliées par deux ! les hymnes  par trois ; nombreuses oraisons, lectures etc.) ; ainsi qu'une flexibilité suffisamment grande pour pouvoir réintroduire un élément aussi antique que les\emph{laudes}.

J'espère que ce petit ouvrage saura faire découvrir la beauté de la prière des emph{laudes} et invitera ses lecteur à une plongée dans l'histoire de la prière chrétienne.

\hfill \textit{Jean-Baptiste Bourgoi, le 7 Novembre 2013.}

\newpage
\SECTION{Les matines de type "cathédrale" en Occident.}

Ce tableau est réalisé à partir des données synthétisées et exposées par Robert Taft dans son livre \emph{La Liturgie des Heures en Orient et en Occident}. Il faut entendre par \emph{matines} ce qu'aujourd'hui nous nommons \emph{Laudes} : la prière du matin.

\begin{center}
\begin{tabular}{ | l | l | l | l | }
\hline
Milan (IV\up{ème} siècle) & Rome (V\up{ème} siècle)    & Gaule (VI\up{ème} siècle)    & Espagne (IV\up{ème} siècle)\\
\hline
Ep 5,14                & \textbf{Ps 148,149,150} & \textbf{Ps 50}            & \textbf{Ps 50}\\
Is 26,9                & \textbf{Ps 50}          & Dn 3                      & Cantique\\
Lecture                & Hymne                   & \textbf{Ps 148,149,150}   & Psaume (dimanche : Dn 3)\\
Oraison                &                         & Capitellum                & \textbf{Ps 148,149,150}\\
Psalmodie              &                         &                           & Hymnes\\
Ps 64,9                &                         &                           & Litanie\\
Ps 118, \textbf{148}   &                         &                           & Oraison\\
                       &                         &                           & Pater\\
                       &                         &                           & Bénédiction\\
\hline
\end{tabular}
\end{center}
   

\SECTION{L'évolution des matines de l'Office Romain.}

Nous reprenons ici aussi les données fournies par Taft au chapitre 7 de son ouvrage \emph{La Liturgie des Heures en Orient et en Occident}.

\textbf{RM} : La Règle du Maître.

\textbf{RC} : Romain classique.

\textbf{RB} : Règle de saint Benoît.

\textbf{PX} : Réforme du bréviaire romain par saint Pie X.

\textbf{LH} : «Liturgia Horarum», la réforme du bréviaire par Paul VI.

\begin{center}
\begin{tabular}{ | l | l | l | l | l | }
\hline
RM (VI\up{ème} siècle)  & RC (V\up{ème} siècle)   & RB (VI\up{ème} siècle)  & PX (1911)              & LH (1970) \\
\hline
                        &                         & \textbf{Ps 66}          &                        & Invit.* (dt. \textbf{Ps 66})\\
                        &                         &                         &                        & Hymne\\
\textbf{Ps 50}          & \textbf{Ps 50}          & \textbf{Ps 50}          & 3 Ps variables         & 1 Ps variable \\
Cantique                & Psaume variable         & 2 Ps variables          &                        & \\
Cantique                & Ps 62,\textbf{66}       & Cantique                & Cantique               & Cantique\\
\textbf{Ps 148,149,150} & \textbf{Ps 148,149,150} & \textbf{Ps 148,149,150} & 1 Ps variable          & 1 Ps variable\\
                        &                         & Lecture                 & Lecture                & Lecture \\
Répons                  &                         & Répons                  &                        & Répons \\
                        &                         & Hymne                   & Hymne           		 &         \\
Verset                  & Verset                  & Verset                  & Verset                 & Verset \\
Épître                  &                         &                         &                        &        \\
Évangile                & Cantique de l'Év.**     & Ctq. de l'Év.**         & Ctq. de l'Év.**        & Ctq. de l'Év.** \\
Répons de l'Év.**       &                         &                         &                        &        \\
\textit{Rogus Dei}***   & \textit{Preces}         & Litanie ****            &                   & \textit{Preces} \\
                        & Notre Père              & Notre Père              &                        & Notre Père \\
                        & ou Oraison              & Oraison                 & Oraison                & Oraison \\
\hline
\end{tabular}
\end{center}

*   : Psaume invitatoire du premier office du jour (Laudes ou Office des Lectures).

**  : Évangile

*** : Peut-être une litanie, ou peut-être une prière silencieuse.

**** : Il s'agit du \emph{Kyrie eleison} qui devait très certainement accompagner des \emph{Preces}.

\newpage
\SECTION{Les Laudes et ses antiennes fériales}

Au sens strict, les \emph{laudes} sont les trois psaumes de louange traditionnels du matin : 148, 149 et 150.

Par extension elles ont progressivement donné leur nom à l'office du matin dans la liturgie romaine.

L'usage des antiennes et des \emph{laudes} (psaumes 148,149,150) doit s'accorder à aux règles présentées dans la \emph{Présentation Générale de la Liturgie des Heures}. 

Il nous faut être attentif aux paragraphes suivant :

\begin{quotation}
\textbf{§246} En certains cas particuliers, on peut choisir dans l'office des formulaires différents de ceux qui se présentent, du moment qu'on ne touche pas à l'organisation générale de chaque heure(\dots)\footnote{\emph{Présentation Générale de la Liturgie des Heures} p.106 in \emph{Prière du Temps Présent}}.
\end{quotation}

\begin{quotation}
\textbf{§247} A l'office des dimanches, des solennités et des fêtes du Seigneur qui figurent au
calendrier universel, des féries du Carême et de la semaine sainte, des jours dans les octaves
de Pâques et de Noël, ainsi qu'aux féries qui vont du 17 au 24 décembre
inclusivement, il n'est jamais permis de changer les formulaires qui sont propres ou appropriés à cette célébration,
comme c'est le cas pour les antiennes, les hymnes, les lectures, les répons, oraisons et même
très souvent les psaumes.

\emph{Aux psaumes dominicaux de la semaine en cours on peut, si on le juge bon, substituer les
psaumes dominicaux d'une autre semaine} et même, s'il s'agit d'un office célébré avec le
peuple, d'autres psaumes choisis pour initier progressivement celui-ci à l'intelligence des
psaumes.
\end{quotation}

\begin{quotation}
\textbf{§252} Bien qu'on doive tenir à l'observation de tout le cycle du psautier réparti par semaines, \emph{si
on le juge bon pour un motif spirituel ou pastoral, au lieu des psaumes assignés à un jour
déterminé, on peut dire des psaumes de la même heure assignés à un autre jour}. Il y a même
des circonstances occasionnelles où il est permis de choisir des psaumes appropriés, ainsi que
d'autres parties, comme pour un office votif
\end{quotation}

Le paragraphe \textbf{246} nous oblige à \textbf{préserver l'organisation générale de chaque heure}.

Le paragraphe \textbf{247} nous permet de \textbf{substituer aux psaumes dominicaux de la semaine en cours ceux d'une autre semaine}.

Les \emph{laudes} sont une suite de psaumes regroupés sous une même antienne, et lus ou chantés sans l'interruption de la doxologie qui n'est dite qu'à la fin du psaume 150. Structurellement ils présentent très clairement comme ne formant qu'un seul psaume.

Les trois psaumes qui forment les \emph{laudes} sont tous des psaumes qui composent le troisième psaume de la prière dominicale matinale de la Liturgie des Heures.

Ainsi, en récitant les \emph{laudes} dans sa forme traditionnelle nous \emph{préservons l'organisation et la structure de l'heure} en ne faisant qu'ajouter \emph{aux psaumes dominicaux de la semaine en cours ceux d'une autre semaine}.

Le paragraphe \textbf{252} de la PGLH nous donne le droit d'utiliser ces \emph{laudes} pour le restant de la semaine.

Il faut cependant faire attention à respecter les exceptions présentées au paragraphe \textbf{247} :

Il est impossible de changer les formulaires propres aux célébrations suivantes :

-- Les antiennes des dimanches.

-- Solennités et fêtes du Seigneur.

-- Féries du Carême.

-- Féries de la Semaine Sainte.

-- Jours dans l'Octave de Pâques.

-- Jours dans l'Octave de Noël.

-- Féries du 17 au 24 décembre inclusivement.

Sachant que les psaumes matinaux des solennités et fêtes sont presque toujours ceux du dimanche I, et que les psaumes des féries du Carême et de Noël, et des jours dans les octaves sont ceux des jours correspondant au psautier, \textbf{nous pouvons dans la plupart des cas, sinon la totalité, utiliser les laudes à la place du troisième psaume matinal de chaque jour}.

%\subsubsection*{Dimanche}
%\gregorien{viii g.}{antiennes/alleluia3x.tex}

\subsubsection*{Aux dimanches, aux fêtes et aux féries en-dehors du temps ordinaire}

Prendre l'antienne proposée dans un livre officiel de l'Office Romain : \textit{Liturgia Horarum} ; \textit{Liturgie des Heures} ; \textit{Prière du Temps Présent} ; \textit{les Heures Grégoriennes} \dots

\subsubsection*{Lundi}
\gregorien{i a.}{antiennes/laudate-dominum.tex}

\subsubsection*{Mardi}
\gregorien{v g.}{antiennes/omnes-angeli.tex}

\subsubsection*{Mercredi}
\gregorien{ii d.}{antiennes/caeli-caelorum.tex}

\subsubsection*{Jeudi}
\gregorien{vi f.}{antiennes/in-sanctis-eius.tex}

\subsubsection*{Vendredi}
\gregorien{i f.}{antiennes/in-tympano.tex}

\subsubsection*{Samedi}
\gregorien{iv e.}{antiennes/in-cymbalis.tex}



\LR{\textbf{Psalmus 148}}{\textbf{Psaume 148}}
\BEGINPARA
\LR{\BEGINENLUL\enluminure{L}{audáte Dóminum} de cælis, * laudáte eum in excélsis.\ENDENLU}
{\BEGINENLUR\enluminure{L}{Louez le Seigneur} du haut des\\ ci\-eux,
louez-le dans les hauteurs.\ENDENLU}
\LR{Laudáte eum, omnes ángeli eius, * laudáte eum, omnes virtútes eius.}
{Vous, tous ses anges, louez-le,
louez-le, tous les univers.}
\LR{Laudáte eum, sol et luna, * laudáte eum, omnes stellæ lucéntes.}
{Louez-le, soleil et lune,
louez-le, tous les astres de lumière ;}
\LR{Laudáte eum, cæli cælórum, * et aquæ omnes, quæ super cælos sunt.}
{vous, cieux des cieux, louez-le,
et les eaux des hauteurs des cieux.}
\LR{Laudent nomen Dómini, * quia ipse mandávit, et creáta sunt;}
{Qu’ils louent le nom du Seigneur :
sur son ordre ils furent créés ;}
\LR{státuit ea in ætérnum et in sæculum sæculi * præcéptum pósuit, et non præteríbit.}
{c’est lui qui les posa pour toujours
sous une loi qui ne passera pas.}
\LR{Laudáte Dóminum de terra, * dracónes et omnes abyssi,}
{Louez le Seigneur depuis la terre,
monstres marins, tous les abîmes ;}
\LR{ignis, grando, nix, fumus, * spíritus procellárum, qui facit verbum eius,}
{feu et grêle, neige et brouillard,
vent d’ouragan qui accomplis sa parole ;}
\LR{montes et omnes colles, * ligna fructífera et omnes cedri,}
{Les montagnes et toutes les collines,
les arbres des vergers, tous les cèdres ;}
\LR{béstiæ et univérsa pécora, * serpéntes et vólucres pennátæ.}
{les bêtes sauvages et tous les troupeaux,
le reptile et l’oiseau qui vole ;}
\LR{Reges terræ et omnes pópuli, * príncipes et omnes iúdices terræ,}
{les rois de la terre et tous les peuples,
les princes et tous les juges de la terre ;}
\LR{iúvenes et vírgines, * senes cum iunióribus,}
{tous les jeunes gens et jeunes filles,
les vieillards comme les enfants.}
\LR{laudent nomen Dómini, * quia exaltátum est nomen eius solíus.}
{Qu’ils louent le nom du Seigneur,
le seul au-dessus de tout nom ;}
\LR{Magnificéntia eius super cælum et terram, * et exaltávit cornu pópuli sui.}
{sur le ciel et sur la terre, sa splendeur : il accroît la vigueur de son peuple.}
\LR{Hymnus ómnibus sanctis eius, * fíliis Israel, pópulo, qui propínquus est ei.}
{Louange de tous ses fidèles,
des fils d’Israël, le peuple de ses proches !}
\LR{\emph{Hic non dicitur Gloria Patri.}}
{\emph{Ici on ne dit pas le Gloire au Père.}}
\LR{\textbf{Psalmus 149}}{\textbf{Psaume 149}}
\LR{\BEGINENLUL\lettrine{C}{antáte Dómino} cánticum novum; * laus eius in ecclésia sanctórum.\ENDENLU}
{\BEGINENLUR\lettrine{C}{hantez au Seigneur} un chant nou\-veau,\\
lou\-ez-le dans l’assemblée de ses fidèles !\ENDENLU}
\LR{Lætétur Israel in eo, qui fecit eum, * et fílii Sion exsúltent in rege suo.}
{En Israël, joie pour son créateur ;
dans Sion, allégresse pour son Roi !}
\LR{Laudent nomen eius in choro, * in tympano et cíthara psallant ei,}
{Dansez à la louange de son nom,
jouez pour lui, tambourins et cithares !}
\LR{quia beneplácitum est Dómino in pópulo suo, * et honorábit mansuétos in salúte.}
{Car le Seigneur aime son peuple,
il donne aux humbles l’éclat de la victoire.}
\LR{Iúbilent sancti in glória, * læténtur in cubílibus suis. }
{Que les fidèles exultent, glorieux,
criant leur joie à l’heure du triomphe.}
\LR{Exaltatiónes Dei in gútture eórum * et gládii ancípites in mánibus eórum,}
{Qu’ils proclament les éloges de Dieu,
tenant en main l’épée à deux tranchants.}
\LR{ad faciéndam vindíctam in natiónibus, * castigatiónes in pópulis, }
{Tirer vengeance des nations,
infliger aux peuples un châtiment,}
\LR{ad alligándos reges eórum in compédibus * et nóbiles eórum in mánicis férreis,}
{charger de chaînes les rois,
jeter les princes dans les fers,}
\LR{ad faciéndum in eis iudícium conscríptum: * glória hæc est ómnibus sanctis eius.}
{leur appliquer la sentence écrite,
c’est la fierté de ses fidèles.}
\LR{\emph{Hic non dicitur Gloria Patri.}}
{\emph{Ici on ne dit pas le Gloire au Père.}}
\LR{\textbf{Psalmus 150}}{\textbf{Psaume 150}}
\LR{\BEGINENLUL\lettrine{L}{audáte} Dóminum in sanctuário eius, * laudáte eum in firmaménto virtútis eius.\ENDENLU}
{\BEGINENLUR\lettrine{L}{ouez} Dieu dans son temple saint, lou\-ez-le au\\ ci\-el de sa puissance ;\ENDENLU}
\LR{Laudáte eum in magnálibus eius, * laudá\-te eum secúndum multitúdinem magnitúdinis eius.}
{louez-le pour ses actions éclatantes, louez-le selon sa grandeur !}
\LR{Laudáte eum in sono tubæ, * lau\-dá\-te eum in psaltério et cíthara,}
{Louez-le en sonnant du cor, louez-le sur la harpe et la cithare ;}
\LR{laudáte eum in tympano et choro, * laudáte eum in chordis et órgano,}
{louez-le par les cordes et les flûtes, louez-le par la danse et le tambour !}
\LR{laudáte eum in cymbalis benesonántibus, † laudáte eum in cymbalis iubilatiónis: * omne quod spirat, laudet Dóminum.}
{Louez-le par les cymbales sonores, louez-le par les cymbales triomphantes ! Et que tout être vivant chante louange au Seigneur !}
\LR{\emph{Glória Patri, et Fílio, * et Spirítui Sancto.}}
{\emph{Gloire au Père, au Fils et au Saint-Esprit.}}
\LR{\emph{Sicut erat in princípio, et nunc et semper, * et in sæcula sæculórum. Amen.}}
{\emph{Au Dieu qui est, qui était et qui vient, pour les siècles des siècles. Amen.}}
\ENDPARA



% Dimanche : Alleluia, alleluia, alleluia.
% Lundi : Laudate dominum de cælis.
% Mardi : Omnes Angeli eius, laudate dominum de cælis.
% Mercredi : Cæli cælorum, laudate Deum.
% Jeudi : In sanctis eius laudate Deum.
% Vendredi : In tympano et choro, in chordis et organo laudate Deum.
% Samedi : In cymbalis benenosantibus laudate Deum.
%\gregorien{.}{antiennes/.tex}

\newpage
\SECTION{Suppléments}

\SUBSECTION{Le psaume invitatoire \& le psaume 50}

Le psaume invitatoire introduit traditionnellement le premier Office du jour qui était auparavant toujours l'Office des Vigiles, et qui est depuis la réforme liturgique issue de Vatican II l'Office des Lectures ou l'Office du Matin.

En explorant l'évolution de l'Office des Heures dans la liturgie catholique nous pouvons constater deux choses :

-- Le psaume 50 a très souvent introduit l'Office du Matin (Laudes) : en Gaule dès le VI\up{ème} siècle, en Espagne dès le IV\up{ème} siècle, dans la Règle du Maître dès le VI\up{ème} siècle et dans l'Office Romain Classique dès le V\up{ème} siècle. Il est présent juste après le Psaume 66 dans la Règle de saint Benoît.

-- Le psaume 66 a fait partie de l'Office Romain à l'Office du Matin dès l'origine, et fut le premier psaume du matin dès saint Benoît.

\subsubsection*{Introduction du psaume 66}

La Liturgia Horarum nous permet de choisir le psaume 66 comme psaume d'invitatoire.

N'hésitons pas, lorsque nous commençons l'Office avec les Laudes, à choisir ce psaume pour lui faire retrouver sa place antique.

\subsubsection*{Introduction du psaume 50}

Il reste le psaume 50. Puisqu'il est le premier psaume du vendredi dans la Liturgia Horarum il est possible d'utiliser la règle du paragraphe 252 de la PGLH pour\textbf{ pouvoir utiliser le psaume 50 à la place du premier psaume du Matin}.

Si nous appliquons ces deux propositions ainsi que la réintroduction des \emph{Laudes}, nous retrouvons pour chaque matin le schéma suivant :

Ps 66\\
Ps 50\\
Cantique\\
Ps 148,149,150\\

Ce qui nous fait retrouver la structure essentielle des "heures cathédrales" romaines de l'Antiquité.

\SUBSECTION{L'oraison dominicale}

Jusqu'à la réforme liturgique issue de Vatican II, l'oraison des laudes, des petites heures et des vêpres était celle du dimanche précédent.

Depuis la réforme, chaque heure propose une oraison spécifique, excepté pour les dimanches et les jours de fêtes.

Les deux manières ont leurs avantages. La manière \emph{ancienne} a le mérite de nous faire méditer l'oraison dominicale pendant toute la semaine, quand la manière \emph{nouvelle} offre une prière adaptée à l'heure, au moment de la journée.

Les paragraphe 246 \& 247 nous permettent d'utiliser les formulaires des collectes du dimanche pour les offices de la férie. Ainsi nous avons le choix entre la manière \emph{ancienne} et la \emph{nouvelle}.

\SUBSECTION{L'hymne}

Avec la réforme liturgique l'Hymne a été déplacée au début de l'office pour les Laudes et les Vêpres. Anciennement, elle se trouvait, pour ces deux heures canoniques, après la lecture et le répons bref (et à sa place actuelle pour les petites heures).

Daniel Saulnier, dans sa présentation de l'Antiphonaire Monastique\footnote{\url{http://palmus.free.fr/Article.pdf}} pour la forme ordinaire du rite, nous apprend que \textit{«les communautés qui souhaitent conserver la place de l’hymne après le répons-bref peuvent continuer à le faire, comme le prévoient les normes du Thesaurus»}.

Peut-être est-il envisageable de pouvoir le faire dans le cadre de l'Office Romain rénové non monastique ?

\end{document}
